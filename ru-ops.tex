\documentclass[11pt,a4]{moderncv}
\usepackage[scale=0.8,top=20mm,bottom=15mm]{geometry}
\usepackage{fontspec}
\usepackage{polyglossia}
\usepackage[unicode]{hyperref}

\setmainfont[Ligatures=TeX,Numbers=OldStyle,Script=Cyrillic]{Noto Serif}
\setsansfont[Ligatures=TeX,Script=Cyrillic]{Noto Sans}
\newfontfamily\bio[Ligatures=TeX,Script=Cyrillic]{Noto Serif} 
\newfontfamily\serif[Ligatures=TeX,Script=Cyrillic]{Noto Serif}
\newfontfamily\sans[Ligatures=TeX,Script=Cyrillic]{Noto Sans}
\newfontfamily\cyrillicfont[Ligatures=TeX,Script=Cyrillic]{Noto Serif}

\renewcommand*{\namefont}{\sans\fontsize{34}{36}\mdseries\upshape}
\renewcommand*{\titlefont}{\serif\LARGE\mdseries\slshape}
\renewcommand*{\addressfont}{\bio\small\mdseries\slshape}
\renewcommand*{\quotefont}{\serif\large\slshape}
\renewcommand*{\sectionfont}{\sans\Large\mdseries\upshape}
\renewcommand*{\subsectionfont}{\sans\large\mdseries\upshape}
\renewcommand*{\hintfont}{}

\moderncvstyle{classic}
\moderncvcolor{grey}


\setmainlanguage{russian}
\setotherlanguage{english}

\name{Демин Тимур Андреевич}{}
\phone{+7 (927) 311-25-68}
\address{}{Уфа}{}
\email{me@tdem.in}
\homepage{tdem.in}
\social[github]{tdemin}
\social[linkedin]{tdemin}
\photo[64pt][0pt]{picture}

\date{\today}
\title{DevOps Engineer}
\hypersetup{
 pdfauthor={Timur Demin},
 pdftitle={DevOps Engineer},
 pdflang={Russian}
}
\begin{document}

\makecvtitle

\section{Образование}
\cventry{\textbf{2017 --- 2021}}{УГНТУ}{}{Бакалавриат, информатика и вычислительная техника}{}{}


\section{Опыт работы}
\cventry{\textbf{дек. 2021 --- июнь 2023}}{DevOps / Platform Engineer}{ООО "РН-БашНИПИнефть"}{}{}{
    \begin{itemize}
    \item
        Сопровождение трех внутренних разрабатываемых систем, их развертывание в тестовые
        и продуктивные контуры.
    \item
        Поддержка внутренней площадки разработки (IDP), изначально основанной на GitLab CE,
        позднее переведенной на Keycloak / Gitea / Jenkins / Nexus / Harbor / SonarQube.
        Поддержка специфичных для IDP компонент: общих библиотек для Jenkins, образов агентов
        Jenkins, библиотеки локальных Docker-образов с преднастроенными внутренними
        репозиториями и удостоверяющими центрами, и т.д. Сопровождение инстанса TeamCity для
        специфичных сборок.
    \item
        Разработка IDP-специфичных утилит сопровождения и автоматизации релизов:
        \begin{itemize}
            \item
                Генератор релизов и списков изменений к ним для Gitea (на Go)
            \item
                Движок групповых политик для Gitea (реализующий настройки по умолчанию
                для новых репозиториев внутри организации, автоматическое создание нужных
                веток, включение защиты веток, и т.д.) (на Go)
            \item
                Утилита массового выполнения операций с репозиториями Gitea (автонастройка
                репозиториев, миграция пользователей и репозиториев с других сервисов, и т.д.)
                (на Python)
        \end{itemize}
    \item
        Поддержка внутренних для IDP (fluent-bit / Loki / Prometheus / Thanos / Alertmanager
        / Grafana) и специфичных для проектов (Logstash / Elasticsearch / Prometheus / Grafana)
        сервисов мониторинга и логгирования. Сопровождение специфичного для команд мониторинга
        (Oracle XE).
    \item
        Сопровождение корпоративного инстанса YouTrack. Разработка рабочих процессов на
        JavaScript для команд, использующих YouTrack для техподдержки (автоопределение организации
        для входящих тикетов из почты, автоназначение команд на задачи, и т.п.)
    \item
        Совмещение ролей DevOps и backend-разработчика на одном из проектов: разработка
        RESTful-сервисов для cron-вебхуков и сборки образов контейнеров на Go / Labstack Echo
        / PostgreSQL / buildah.
    \item
        Техническая поддержка разработчиков в случае необходимости.
    \item
        Сопровождение инфраструктуры и документации к ней как Ansible / helmfile-кода
        там, где это возможно. Ведение локальной библиотеки Ansible-ролей и модулей
        для административных задач.
    \end{itemize}
}
\cventry{\textbf{2019 --- н.в.}}{Сопровождающий}{tdem.in}{}{}{
    Сопровождение различных веб-сервисов (Nextcloud, TTRSS, Gitea, Drone) для
    личного пользования. Настройка автоматической сборки Docker-образов с Gitea / Drone CI
    (позднее замена на GitHub Actions) для развертывания на персональном сервере.
    Конфигурация автоматического резервного копирования. Разработка модуля GitHub Actions
    для автоматизированного забора Git-тегов разнородного ПО для автоматических сборок.
}

\section{Навыки}
\cvitem{\bfseries Общее}{Linux (Ubuntu, Debian, Red Hat), Git, сети}
\cvitem{\bfseries Контейнеры}{Docker, rootless podman / buildah}
\cvitem{\bfseries CI/CD}{Jenkins, GitHub Actions, Drone CI (1.x), GitLab CI, TeamCity}
\cvitem{\bfseries Управление конфигурацией, IaaC}{Ansible, helmfile}
\cvitem{\bfseries Логгирование}{fluent-bit, Logstash, Loki}
\cvitem{\bfseries Мониторинг}{Prometheus, Thanos, Grafana, Alertmanager}
\cvitem{\bfseries Оркестрация}{Kubernetes (bare metal, k3s), docker-compose}
\cvitem{\bfseries СУБД}{PostgreSQL, Redis, Elasticsearch}
\cvitem{\bfseries Резервное копирование}{restic}
\cvitem{\bfseries Разработка}{Go, Python, bash, Groovy}

\section{Языки}
\cvitem{\bfseries Русский}{родной}
\cvitem{\bfseries Английский}{свободное владение}


\end{document}
