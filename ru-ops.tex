\documentclass[10pt,a4]{moderncv}
\usepackage[scale=0.9,top=5mm,bottom=5mm,left=5mm,right=5mm]{geometry}
\usepackage{fontspec}
\usepackage{polyglossia}
\usepackage[unicode]{hyperref}
\recomputelengths
\setmainfont[Ligatures=TeX,Numbers=OldStyle,Script=Cyrillic]{DejaVu Serif}
\setsansfont[Ligatures=TeX,Script=Cyrillic]{DejaVu Sans}
\newfontfamily\bio[Ligatures=TeX,Script=Cyrillic]{DejaVu Serif}
\newfontfamily\serif[Ligatures=TeX,Script=Cyrillic]{DejaVu Serif}
\newfontfamily\sans[Ligatures=TeX,Script=Cyrillic]{DejaVu Sans}
\newfontfamily\cyrillicfont[Ligatures=TeX,Script=Cyrillic]{DejaVu Serif}
\renewcommand*{\namefont}{\sans\fontsize{24}{28}\mdseries\upshape}
\renewcommand*{\titlefont}{\serif\LARGE\mdseries\slshape}
\renewcommand*{\addressfont}{\bio\small\mdseries\slshape}
\renewcommand*{\quotefont}{\serif\large\slshape}
\renewcommand*{\sectionfont}{\sans\Large\mdseries\upshape}
\renewcommand*{\subsectionfont}{\sans\large\mdseries\upshape}
\renewcommand*{\hintfont}{}
\moderncvstyle{classic}
\moderncvcolor{grey}

\setmainlanguage{russian}
\setotherlanguage{english}

\name{Демин Тимур Андреевич}{}
\phone{+7 (771) 317-26-67}
\address{}{Астана}{Казахстан}
\email{me@tdem.in}
\homepage{tdem.in}
\social[github]{tdemin}
\social[linkedin]{tdemin}
\photo[96pt][0.4pt]{picture}

\date{\today}
\title{DevOps Engineer}
\hypersetup{
 pdfauthor={Timur Demin},
 pdftitle={DevOps Engineer},
 pdflang={Russian}
}
\begin{document}

\makecvtitle
\vspace*{-5mm}

\section{Навыки}
\cvitem{\bfseries Общее}{Linux (Ubuntu, Debian, Red Hat), Git, сети}
\cvitem{\bfseries Оркестрация}{Kubernetes (bare metal, k3s, RKE2, vanilla Kubernetes), docker-compose}
\cvitem{\bfseries Контейнеры}{Docker, Buildkit, rootless podman / buildah}
\cvitem{\bfseries CI}{GitLab CI, Jenkins, GitHub Actions, Drone CI (1.x), TeamCity}
\cvitem{\bfseries CD}{Argo CD, Flux CD (2.x)}
\cvitem{\bfseries IaC}{Ansible, kustomize, Terraform, helmfile}
\cvitem{\bfseries Логгирование}{Loki, fluent-bit, Logstash}
\cvitem{\bfseries Мониторинг}{Prometheus, Grafana, Alertmanager}
\cvitem{\bfseries СУБД}{PostgreSQL / Patroni, Redis, Elasticsearch}
\cvitem{\bfseries Разработка}{Go (Echo, Gorilla), Python, bash, JavaScript}
\cvitem{\bfseries Разное}{restic, MinIO, Sonatype Nexus, Harbor, Keycloak}

\section{Образование}
\cventry{\textbf{2017 --- 2021}}{УГНТУ}{}{Бакалавриат, информатика и вычислительная техника}{}{}

\section{Языки}
\cvitem{\bfseries Русский}{родной}
\cvitem{\bfseries Английский}{свободное владение}

\section{Опыт работы}
\cventry{\textbf{май 2025 --- по н.в.}}{DevOps Engineer}{Aventus IT}{}{}{}
\cventry{\textbf{сен. 2023 --- май 2025}}{DevOps / Service Reliability Engineer}{АО "Центр электронных финансов"}{}{}{
    Ключевые технологии: Kubernetes, Go, Argo CD, GitLab CI, Flux CD
}
\cventry{\textbf{дек. 2021 --- июнь 2023}}{DevOps / Platform Engineer}{ООО "РН-БашНИПИнефть"}{}{}{
    \begin{itemize}
    \item
        Сопровождение инфраструктуры в Docker и Kubernetes, документации к ней как Ansible /
        helmfile-кода там, где это возможно. Ведение локальной библиотеки Ansible-ролей и модулей
        для административных задач.
    \item
        Сопровождение внутренних разрабатываемых систем, их развертывание в тестовые и
        продуктивные контуры. Поддержка пайплайнов CI/CD для них на Jenkins.
    \item
        Поддержка внутренней площадки разработки (IDP), изначально основанной на GitLab CE,
        позднее переведенной на Keycloak / Gitea / Jenkins / Nexus / Harbor / SonarQube.
        Поддержка специфичных для IDP компонент: общих библиотек для Jenkins, образов агентов
        Jenkins, библиотеки локальных Docker-образов с преднастроенными внутренними
        репозиториями и удостоверяющими центрами, и т.д. Сопровождение инстанса TeamCity для
        специфичных сборок.
    \item
        Разработка IDP-специфичных утилит сопровождения и автоматизации релизов:
        \begin{itemize}
            \item
                Генератор релизов и списков изменений к ним для Gitea (на Go)
            \item
                Движок групповых политик для Gitea (реализующий настройки по умолчанию
                для новых репозиториев внутри организации, автоматическое создание нужных
                веток, включение защиты веток, и т.д.) (на Go)
            \item
                Утилита массового выполнения операций с репозиториями Gitea (автонастройка
                репозиториев, миграция пользователей и репозиториев с других сервисов, и т.д.)
                (на Python)
        \end{itemize}
    \item
        Поддержка внутренних для IDP (fluent-bit / Loki / Prometheus / Thanos / Alertmanager
        / Grafana) и специфичных для проектов (Logstash / Elasticsearch / Prometheus / Grafana)
        сервисов мониторинга и логгирования. Сопровождение специфичного для команд мониторинга
        (Oracle XE).
    \item
        Сопровождение корпоративного инстанса YouTrack. Разработка рабочих процессов на
        JavaScript для команд, использующих YouTrack для техподдержки (автоопределение организации
        для входящих тикетов из почты, автоназначение команд на задачи, и т.п.)
    \item
        Совмещение ролей DevOps и backend-разработчика на одном из проектов: разработка
        RESTful-сервисов для cron-вебхуков и сборки образов контейнеров на Go / Labstack Echo
        / PostgreSQL / buildah.
    \item
        Техническая поддержка разработчиков в случае необходимости.
    \end{itemize}
}
\cventry{\textbf{июнь 2020 --- июль 2020}}{Практикант}{АО "Нефтеавтоматика"}{}{}{
    \begin{itemize}
    \item
        Разработка внутреннего ПО учета микроконтроллеров на Go / Gorilla / Pongo2 /
        PostgreSQL. Тестирование и отладка на Debian 9.
    \end{itemize}
}

\section{О себе}
Поддерживаю движение свободного программного обеспечения. Бывший корректор новостей на
портале Linux.org.ru. Сопровождающий пакетов в Arch Linux User Repository и nixpkgs.
Активно веду проекты на GitHub, интересуюсь backend и web-разработкой в целом. Особо
интересуюсь практической информационной безопасностью, а также изоляцией, контейнеризацией
и защитой приложений на уровне сети и операционных систем.

\end{document}
