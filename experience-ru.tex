\section{Опыт работы}
\cventry{\textbf{сен. 2023 --- по н.в.}}{DevOps Engineer}{АО "Центр электронных финансов"}{}{}{
    Ключевые технологии: Kubernetes, Go, Argo CD, GitLab CI, Flux CD
}
\cventry{\textbf{дек. 2021 --- июнь 2023}}{DevOps / Platform Engineer}{ООО "РН-БашНИПИнефть"}{}{}{
    \begin{itemize}
    \item
        Сопровождение инфраструктуры в Docker и Kubernetes, документации к ней как Ansible /
        helmfile-кода там, где это возможно. Ведение локальной библиотеки Ansible-ролей и модулей
        для административных задач.
    \item
        Сопровождение внутренних разрабатываемых систем, их развертывание в тестовые и
        продуктивные контуры. Поддержка пайплайнов CI/CD для них на Jenkins.
    \item
        Поддержка внутренней площадки разработки (IDP), изначально основанной на GitLab CE,
        позднее переведенной на Keycloak / Gitea / Jenkins / Nexus / Harbor / SonarQube.
        Поддержка специфичных для IDP компонент: общих библиотек для Jenkins, образов агентов
        Jenkins, библиотеки локальных Docker-образов с преднастроенными внутренними
        репозиториями и удостоверяющими центрами, и т.д. Сопровождение инстанса TeamCity для
        специфичных сборок.
    \item
        Разработка IDP-специфичных утилит сопровождения и автоматизации релизов:
        \begin{itemize}
            \item
                Генератор релизов и списков изменений к ним для Gitea (на Go)
            \item
                Движок групповых политик для Gitea (реализующий настройки по умолчанию
                для новых репозиториев внутри организации, автоматическое создание нужных
                веток, включение защиты веток, и т.д.) (на Go)
            \item
                Утилита массового выполнения операций с репозиториями Gitea (автонастройка
                репозиториев, миграция пользователей и репозиториев с других сервисов, и т.д.)
                (на Python)
        \end{itemize}
    \item
        Поддержка внутренних для IDP (fluent-bit / Loki / Prometheus / Thanos / Alertmanager
        / Grafana) и специфичных для проектов (Logstash / Elasticsearch / Prometheus / Grafana)
        сервисов мониторинга и логгирования. Сопровождение специфичного для команд мониторинга
        (Oracle XE).
    \item
        Сопровождение корпоративного инстанса YouTrack. Разработка рабочих процессов на
        JavaScript для команд, использующих YouTrack для техподдержки (автоопределение организации
        для входящих тикетов из почты, автоназначение команд на задачи, и т.п.)
    \item
        Совмещение ролей DevOps и backend-разработчика на одном из проектов: разработка
        RESTful-сервисов для cron-вебхуков и сборки образов контейнеров на Go / Labstack Echo
        / PostgreSQL / buildah.
    \item
        Техническая поддержка разработчиков в случае необходимости.
    \end{itemize}
}
\cventry{\textbf{июнь 2020 --- июль 2020}}{Практикант}{АО "Нефтеавтоматика"}{}{}{
    \begin{itemize}
    \item
        Разработка внутреннего ПО учета микроконтроллеров на Go / Gorilla / Pongo2 /
        PostgreSQL. Тестирование и отладка на Debian 9.
    \end{itemize}
}
\cventry{\textbf{2019 --- н.в.}}{Сопровождающий}{tdem.in}{}{}{
    Сопровождение различных веб-сервисов (Nextcloud, TTRSS, Gitea, Drone) для
    личного пользования. Настройка автоматической сборки Docker-образов с Gitea / Drone CI
    (позднее замена на GitHub Actions) для развертывания на персональном сервере.
    Конфигурация автоматического резервного копирования. Разработка модуля GitHub Actions
    для автоматизированного забора Git-тегов разнородного ПО для автоматических сборок.
}
